% !TEX program = pdflatex
\documentclass{tufte-handout}

\title{\centering A classical introduction to modern number theory}
\author{I'm the author}


\date{\today} % without \date command, current date is supplied

%\geometry{showframe} % display margins for debugging page layout

\usepackage{graphicx} % allow embedded images
  \setkeys{Gin}{width=\linewidth,totalheight=\textheight,keepaspectratio}
\usepackage{amsmath}  % extended mathematics
\usepackage{amssymb}  % extended math symbol collection
\usepackage{booktabs} % book-quality tables
\usepackage{units}    % non-stacked fractions and better unit spacing
\usepackage{multicol} % multiple column layout facilities
\usepackage{lipsum}   % filler text
\usepackage{fancyvrb} % extended verbatim environments
  \fvset{fontsize=\normalsize}% default font size for fancy-verbatim environments

% Standardize command font styles and environments
\newcommand{\doccmd}[1]{\texttt{\textbackslash#1}}% command name -- adds backslash automatically
\newcommand{\docopt}[1]{\ensuremath{\langle}\textrm{\textit{#1}}\ensuremath{\rangle}}% optional command argument
\newcommand{\docarg}[1]{\textrm{\textit{#1}}}% (required) command argument
\newcommand{\docenv}[1]{\textsf{#1}}% environment name
\newcommand{\docpkg}[1]{\texttt{#1}}% package name
\newcommand{\doccls}[1]{\texttt{#1}}% document class name
\newcommand{\docclsopt}[1]{\texttt{#1}}% document class option name
\newenvironment{docspec}{\begin{quote}\noindent}{\end{quote}}% command specification environment

% Standardize command for math 
\newcommand{\Z}{\mathbb{Z}}
\newcommand{\C}{\mathbb{C}}
\newcommand{\R}{\mathbb{R}}

\newcounter{mark}

%%%%%%%%%%%%%%%%%%%%%%%%%%%%%%%%%%%%%%%%%%%%%%%%%%%%%%%%%%%%%%%%%%%%%%%%%%%%%%%%%%%%%%%%%%%%%%%%%%%%%%
% add numbers to chapters, sections, subsections
\setcounter{secnumdepth}{2}
\usepackage{xcolor}
\definecolor{g1}{HTML}{077358}
\definecolor{g2}{HTML}{00b096}
% chapter format  %(if you use tufte-book class)
%\titleformat{\chapter}%
%{\huge\rmfamily\itshape\color{red}}% format applied to label+text
%{\llap{\colorbox{red}{\parbox{1.5cm}{\hfill\itshape\huge\color{white}\thechapter}}}}% label
%{2pt}% horizontal separation between label and title body
%{}% before the title body
%[]% after the title body

% section format
\titleformat{\section}%
{\normalfont\Large\itshape\color{g1}}% format applied to label+text
{\llap{\colorbox{g1}{\parbox{1.5cm}{\hfill\color{white}\thesection}}}}% label
{1em}% horizontal separation between label and title body
{}% before the title body
[]% after the title body

% subsection format
\titleformat{\subsection}%
{\normalfont\large\itshape\color{g2}}% format applied to label+text
{\llap{\colorbox{g2}{\parbox{1.5cm}{\hfill\color{white}\thesubsection}}}}% label
{1em}% horizontal separation between label and title body
{}% before the title body
[]% after the title body

%%%%%%%%%%%%%%%%%%%%%%%%%%%%%%%%%%%%%%%%%%%%%%%%%%%%%%%%%%%%%%%%%%%%%%%%%%%%%%%%%%%%%%%%%%%%%%%%%%%%%%
\usepackage{color-tufte}
%%%%%%%%%%%%%%%%%%%%%%%%%%%%%%%%%%%%%%%%%%%%%%%%%%%%%%%%%%%%%%%%%%%%%%%%%%%%%%%%%%%%%%%%%%%%%%%%


\begin{document}

\maketitle% this prints the handout title, author, and date

\begin{abstract}
\noindent
A simple notes template. Inspired by Tufte-\LaTeX class and beautiful notes by \begin{verbatim*}
	https://github.com/abrandenberger/course-notes
\end{verbatim*}
\end{abstract}

%\printclassoptions

\section{Groups}\label{sec:groups}

\subsection{Laws of Composition}


\begin{problem}
	\normalfont
	Let $a, b\in S$, assume operation of S is associative, and its identity is $e$. If $a$ is left inverse of $b$, does this imply that $a$ is right inverse of $b$?
\end{problem}

\begin{proof}
	Suppose $b$ has left inverse $a$ and right inverse $c$: $ab = e, bc = e$ but $a\neq c$. Then $ae = a = a(bc)=(ab)c=c$, which is a contradiction.
\end{proof}

\begin{enumerate}
	\item If $la=e,ar=e$ (it imply that a has both left and right inverse), then $l=r$.
	\item If $a$ is invertible, its inverse is unique.
	\item Inverse multipy in the opposite order: $(ab)^{-1}=b^{-1}a^{-1}$
	\item An element $a$ may have a left inverse or a right inverse, though it is not invertible.
\end{enumerate}

The last statement is unique and interesting.

\marginnote[1cm]{Consider how to prove this lemma.}
\begin{lemma}
	\normalfont
	Every nonzero integer can be written as a product of primes.
\end{lemma}

\marginnote[1cm]{Easy to prove.}
\begin{lemma}
	If $a,b \in \Z$ and $b>0$, there exist $q,r\in \Z$ such that $a=qb+r$ with $0\le r < b$.
\end{lemma}

\marginnote[1cm]{We often see $(a,b)=d$, it means $(a,b)=(d)$ in fact.}

\begin{definition}
\lipsum[1][0-10]
\end{definition}

\begin{definition}%%  [can be kept empty]
	\normalfont
	Here's is the beautiful Schr\"odinger equation
	\[ i\hbar {\frac {\partial }{\partial t}}\Psi (x,t)=
	\left[-{\frac {\hbar ^{2}}{2m}}{\frac {\partial ^{2}}{\partial x^{2}}}+V(x,t)\right]\Psi (x,t)\]
\end{definition}

\subsection{Groups and Subgroups}

A group is a set $G$ together with a law of composition that has the following properties:

\begin{enumerate}
	\item associative, $(ab)c=a(bc)$ for all $a,b,c\in G$
	\item identity element $e$, $ea=ae=a$ for all $a\in G$
	\item for all $a\in G$, $a$ has a inverse $b$, such that $ab=ba=1$
\end{enumerate}

An $abelian group$ is a group whose law of composition is commutative. For example, the set of nonzero real numbers forms an abelian group under multiplication, and the set of all real numbers forms a abelian group under addition.

\begin{proposition}[\textbf{Cancellation Law}]
	Let $a,b,c$ be elements of a group $G$ whose law of composition is written multiplicatively. If $ab=ac$ or if $ba=ca$, then $b=c$. If $ab=a$ or if $ba=a$, then b=1.
\end{proposition}
\begin{proof}
	Multipy both sides of $ab=ac$ on the left by $a^{-1}$ to obtain $b=c$. The other proofs are analogous. 
\end{proof}

\begin{enumerate}
	
	\item The $n\times n$ general linear group is the group of all invertible $n\times n$ matrices. It is denoted by $GL_n = {n\times n~\text{invertible matrices A}}$. $GL_n(\mathbb{R}), GL_n(\mathbb{C})$ indicate matrices units are real or complex number. If all matrices of the group have determinant 1, then it's called the special linear group, it's a subgroup of $GL_n$, it's denoted by $SL_n$.
	\item $S_n$ is the group of permutations of $\{1, 2,\cdots,n\}$, sometimes it's called the symmetric group. The symmetric group $S_n$ is a finite group of order $n!$.
\end{enumerate}

The permutations of a set ${a, b}$ of two elements are the identity and the transposition. It's a group of order two. Notice the difference between this set and $S_2$, especially definition of $S_n$.

Every group $G$ has two obvious subgroups: the group $G$ itself, and the trivial subgroup that consists of the identity element alone.

\subsection{Subgroups of the Additive Group of Integers}

Let $a$ be an integer different from 0. We denote the subset of $\mathbb{Z}$ that consists of all multiples of $a$ by $\mathbb{Z}a$: 
\begin{equation}
	\Z a =\{n\in Z\mid n=ka~\text{for some k in } \Z\}.
\end{equation}

\begin{theorem}
	\normalfont
	Let $S$ be a subgroup of additive group $\Z^{+}$ $(\mathrm{or}~(\Z, +))$. Either $S$ is the trivial subgroup ${0}$, or else it has the form $\Z a$, where $a$ is the smallest positive integer in $S$.
\end{theorem}

$Za\cap Zb=Zm, m=\mathrm{lcm}(a, b)$, and $Za + Zb=Za\cup Zb=Zn, n = \gcd(a, b)$.

\subsection{Cycle Groups}

A group is called cyclic if there exists a $g\in G$ such that $G=\{g^k\mid k\in \Z\}$.

$\langle x \rangle$ is a cyclic subgroup of a group $G$,

\begin{proposition}
	\normalfont
	Let $x$ be an element of finite order $n$ in a group, and let $k$ be an integer that is written as $k=nq$$+r$ where $q$ and $r$ are integers and $r$ is in the range $0\le r<n$.
	\begin{enumerate}
		\item $x^k=x^r$.
		\item $x^k=1$ if and only if $r=0$.
		\item Let $d=(k,n)$, the order of $x^k$ is equal to $n/d$.
	\end{enumerate}
\end{proposition}

Notice the difference between order of $x$ and $x^k$.

\subsection{Homomorphisms}

Let $G$ and $G'$ be groups, written with multiplicative notation. A \textbf{homomorphism} $\phi: G\rightarrow G'$ is a map from $G$ to $G'$ such that for all $a$ and $b$ in $G$
\begin{equation}
	\phi(ab)=\phi(a)\phi(b)
	\nonumber
\end{equation}

Intuitively, a homomorhisms is a map that is compatible with the laws of composition in the two groups, and it provides a way to relate different groups, in brief, it's a map from one algebra to another, such as from one group to another.

There are many homomorphism examples, such as the absolute value map $||: (\C, \times)\rightarrow (\R, \times)$, the determinant function det: $GL_n(\R)\rightarrow (\R, \times)$.

\begin{proposition}
	\normalfont
	Let $\phi : G\rightarrow G'$ be a group homomorphism.
	\begin{enumerate}
		\item If $a_1,\cdots,c_k \in G$, then $\phi(a_1\cdots a_k)=\phi(a_1)\cdots\phi(a_k)$.
		\item $\phi$ maps the identity to the identity: $\phi(e_{G})=e_{G}$.
		\item $\phi$ maps the inverse to inverse: $\phi(a^{-1})=\phi(a)^{-1}$.
	\end{enumerate}
\end{proposition}

\begin{definition}
	\normalfont
	The image of homomorphism $\rho: G\rightarrow H$ is the set $\{\rho(g)\mid g\in G\}\subset H$, written as $\rho(G)$, the kernel of $\rho$ is the set $\{g\mid rho(g)=e_{H}\}$, written as $\rho(g)^{-1}$.
\end{definition}

So $\rho(g)^{-1}$ is the set of all $g\in G$ maped to identity of $H$. The $\rho(G)$ is a subgroup of $H$, and $\rho(e_{H})^{-1}$ is a subgroup of $G$. Notice that the kernel of a homomorphism might contain multiple elements. The identity of $G$ must be maped to the identity of $H$, but not only the identity of $G$ is maped to the identity of $H$. Such as homomorphism $\rho: \Z_6\rightarrow \Z_3$, $\rho(0,\cdots,5)=0,1,2,0,1,2$, so image of $\rho$ is $\Z_3$, and kernel of $rho$ is {0,3}. Another example is $\rho: \Z_3\rightarrow \Z_6, \rho(n)=2n$, so the image is $\{0,2,4\}$, and again, the kernel is just 0. 

\textbf{left coset}: If $H$ is a subgroup of group $G$, a is in $G$, then 

\begin{equation}
	aH = \{ah\mid h\in H\}
\end{equation}

\begin{proposition}
	\normalfont
	Let $\phi: G\rightarrow G'$ be a homomorphism of groups, and let $a, b\in G$. Let $K$ be the kernel of $\phi$. The following four statement are equivalent:
	\begin{enumerate}
		\item $\phi(a)=\phi(b)$
		\item $a^{-1}b$ is in $K$
		\item $b$ is in the coset of $aK$.
		\item The coset $bK$ and $aK$ are equal.
	\end{enumerate}
\end{proposition}

\begin{corollary}
	\normalfont
	A homomorphism $\phi: G\rightarrow G'$ is injective if and only if its kernel $K$ is the trivial subgroup \{1\} of $G$.
\end{corollary}

If $a$ and $g$ are elements of a group $G$, the element $gag^{-1}$ is called the conjugate of $a$ by $g$.

\begin{definition}
	\normalfont
	A subgroup $N$ of a group $G$ is a normal subgroup if for every $a$ in $N$ and every $g$ in $G$, the conjugate $gag^{-1}$ is in $N$.
\end{definition}

\begin{proposition}
	\normalfont
	The kernel of a homomorphism is a normal subgroup.
\end{proposition}

\begin{proof}
	If $a$ is in the kernel of a homomorphism $\phi: G\rightarrow G'$ and if any element of $G$, then $\phi(gag^{-1})=\phi(g)\phi(a)\phi(g^{-1})=\phi(g)1\phi(g)^{-1}=1$, therefore $gag^{-1}$ is in the kernel too. So the kernel of a homomorphism is normal.
\end{proof}

Thus the special linear group $SL_n(\R)$ is a normal subgroup of the general linear group $GL_n(\R)$. And every subgroup of an abelian group is normal.

The center of a group $G$, which is often denoted by $Z$, is the set:

\begin{equation}
	Z=\{z\mid  zx =zx,z\in G, \text{for all }x \in G\}
\end{equation}

It's always a normal subgroup of $G$. The \textbf{center} of the special linear group $SL_2(\R)$ consists of the two matrices $I,-I$.

\subsection{Headings}\label{sec:headings}
\marginnote{\begin{proof}[Proof (Theorem 1.1)] 
		
		\lipsum[1][1-3]\end{proof}}
\begin{theorem}%%  [can be kept empty]
	\lipsum[1][1-3] %% for dummy text
\end{theorem}

\begin{lemma}%%  [can be kept empty]
	\lipsum[1][1-3] %% for dummy text
	
\end{lemma}
\begin{proof}
	\lipsum[1][1-5]
\end{proof}

%\marginnote{\begin{proof}\lipsum[1][1-3]\end{proof}}

\begin{corollary}%%  [can be kept empty]
	\lipsum[1][1-3] %% for dummy text
\end{corollary}

\begin{proposition}
	\lipsum[1][1-3] %% for dummy text
\end{proposition}
\begin{problem}
	\lipsum[1][1-2]
\end{problem}

\begin{proof}
	\lipsum*[1]
\end{proof}


\end{document}