\documentclass{article}

% Language setting
% Replace `english' with e.g. `spanish' to change the document language
\usepackage[UTF8]{ctex}
% Set page size and margins
% Replace `letterpaper' with `a4paper' for UK/EU standard size
\usepackage[letterpaper,top=2cm,bottom=2cm,left=3cm,right=3cm,marginparwidth=1.75cm]{geometry}

% Useful packages
\usepackage{amssymb}
\usepackage{amsmath}
\usepackage{graphicx}
\usepackage{hyperref}
\usepackage{color}

\renewcommand{\theequation}{\arabic{section}.\arabic{equation}}
\newcommand{\setParDis}{\setlength {\parskip} {0.3cm} }
\newcommand{\nident}{\setlength{\parindent}{0pt}}



% \title{Your Paper}
% \author{You}

\begin{document}

\setParDis
% \nident

\section{素数}

\textbf{1.} 形如 $M_p=2^p-1$ 的数称为梅森数,若$M_p$为素数,那么称之为梅森素数。

\begin{itemize}
    \item 若$n$为合数,则$M_n$为合数,若$M_p$为素数,则$p$为素数
    \item $n$为大于 1 的奇数时,$M_n$的所有因子形式为$8k-1$或$8k+1$(2 一定是$p$的二次剩余,可以用 Legendre 符号证明)
    \item $p$为奇素数时,$M_p$的所有素因子可以表示为$2kp+1$的所有因子形式
    \item[] 证明:令$q$为$M_p$的任意素因子,由费马小定理得$q\mid 2^{q-1}-1$。所以$q\mid (2^{q-1},2^p-1)=2^{(q-1,p)}-1$,若$(q-1,p)=1$,显然不可能,所以$(q-1,p)=p$,所以$p\mid q-1$,又$q-1$为偶数,所以存在$k$,使$q=2kp+1$
\end{itemize}

\textbf{2.} 若$n>1$且$a^n-1$为素数,则$a=2$,且$n$为素数

\textbf{3.} 若$2^m+1$为素数,则$m=2^n$。($F_n=2^{2^n}+1$ 形式的数叫做费马数,它也不一定是素数)


\section{余数}

定义

$$\mathbb{Z}/n\mathbb{Z}\ \text{or} \ \mathbb{Z}_n=\{[a]_n: 0\le a\le n-1\}$$

\textbf{1.} $[a]_n$中有一个元素与$n$互素,那么$[a]_n$所有元素与$n$互素;如果$n$是一个素数,那么它的所有的剩余类都与$n$互素

\textbf{2.} $a$ 在模 $n$ 情况下的逆元存在当且仅当$gcd(a, n)=1$,所以$n$是一个素数,$1$到$n-1$之间的所有数都有逆元;正常情况下有$\phi(n)$个存在逆元的数;

\textbf{3.} $\mathbb{Z}_n$ 是一个域当且仅当 $n$是一个素数;

\textbf{4.} $S$是一个完系的充要条件:

\begin{itemize}
    \item $S$包含$n$个元素
    \item $S$中任意两个元素不同余
\end{itemize}


\textbf{5.} 若 $a\equiv b\pmod m$,且 $\gcd(k,m)=1$,$k$ 是一个正整数,那么 $ka\equiv kb\pmod m$,更一般的,$ak\equiv bk\pmod m \Leftrightarrow a \equiv b\pmod{\frac{m}{\gcd(k,m)}}$

\textbf{6.} $a\equiv b\pmod{m}$且$a\equiv b\pmod{n}\Leftrightarrow a\equiv b\pmod{lcm(m,n)}$。这个又两种方法证明,第一种唯一素因子分解,第二种$a\equiv b\pmod{m}$等价于$m\mid a-b$,对于$n、lcm(m,n)$同理。那么,对于一个$a\equiv b\pmod{m}$,我们就可以把这个拆成模$m$的所有的素因子的幂的同余式的组。

\subsection{完系}

\textbf{1.} 若 $a_i (1\le n)$ 是一个完系,$k, m\in \mathbb{Z}, (m, n)=1$,则$k + ma_i(1\le n)$ 也构成一个模 $n$ 的完系

\textbf{2.} 若 $a_i (1\le n)$ 是一个完系,则$\sum_{i\le n}a_i=n(n+1)/2\pmod n$,右边这个结果最终要么是 $n/2$,要么是0

\subsection{缩系}

$n$ 是正整数,$S$ 是缩系的充要条件是

\begin{itemize}
    \item S包含$\varphi (n)$个元素
    \item S中任意两个元素不同余
    \item S中任意元素与n互素
\end{itemize}


\textbf{1.} 设 $a_1,a_2,\cdots,a_{\phi(m)}$ 为一缩系,且$\gcd(m, k)=1$,那么$ka_1,ka_2,\cdots,ka_{\phi(m)}$也组成一个缩系。否则若$ka_i\equiv ka_j\pmod m$,因为$\gcd(m, k)=1$,所以$a_i\equiv a_j\pmod m$,矛盾。

\subsection{线性同余方程}

形如$ax \equiv c \pmod b$的方程被称为线性同余方程(Congruence Equation)。

\textbf{1.} 方程$ax+by=c$ 与方程 $ax \equiv c \pmod b$ 是等价的,有整数解的充要条件为 $\gcd(a,b) \mid c$。

\textbf{2.} 若$d=\gcd(a,b)$,且$d\mid c$, $t$为原方程的解,则所有的解可以表示为$\{t+k\dfrac{n}{d}\mid k\in \mathbb{Z}\}$,也是$\dfrac{a}{d}x\equiv \dfrac{b}{d}\pmod{\dfrac{n}{d}}$的解。所有这些解模$n$最后会落到$n$的完全剩余系的$d$个元素之上。$t, t+n/d, t+2n/d, \cdots, t+(d-1)n/d$就是$d$个这样的解。其中最小正整数解为$x=(x \bmod t+t) \bmod t$,其中 $t=\dfrac{b}{\gcd(a,b)}$。

\subsection{Euler's}

\textbf{1.} 若 $(k,m)=1$,则 $k^{\phi(m)}\equiv 1\pmod m$,证明:由 $\prod_{i=1}^{\phi(m)}(ka_i)\equiv \prod_{i=1}^{\phi(m)}a_i\pmod m$,由$(a_i,m)=1$,那么$k^{\phi(m)}\equiv 1\pmod m$,证毕。

\textbf{2. 费马小定理:}若$p$为素数,那么对所有整数$a$有,$a^p\equiv a\pmod p$,或$a^{p-1}\equiv 1\pmod p$

\textbf{3.} $p$ 是一个素数,$\phi(p^k)=p^k-p^{k-1}$,证明先考虑 0 到 $p^k-1$,有多少$p$的倍数,然后减一下

\textbf{4.} 任意整数 $n$,有 $\sum_{d\mid n}\phi(n)=n$,因为 $n=\sum_{d\mid n}\phi(n/d)=\sum_{d\mid n}\phi(d)$,$\phi(n/d)$ 表示与 $n$ 的最大公因数为 $d$ 的数的个数,关于 $n=\sum_{d\mid n}\phi(n/d)$,我们想要说明它刚好包括所有 $n$ 个数,因为没有任意一个数字使得 $gcd(n, a)=d_1,gcd(n,a)=d_2$ 使得这个数字进入两个不同的划分集合 $d_1, d_2$,所有可以恰好包括所有数字。还需要说明 1 到 $n$ 内的任意一个数字都与 $n$ 有一个公共因子,比较显然,最起码是 1

\textbf{5.} $\not\equiv$

\textbf{note: }欧拉定理要求$a、p$必须互素,注意到使用缩系证明过程中,给一个缩系乘以一个与$p$互素的数$a$,得到的集合依然是一个缩系,如果与$p$不互素,显然就会出现问题。而费马小定理中要求$p$是一个素数,更加严格。


\subsection{原根}

次数为$\varphi(p)$的数称为$p$的原根,一个推论是次数为$p-1$的数,一定是$p$的原根($d\mid \varphi(p)\le p-1$

\textbf{1.} $a$模$m$的次数是$l$,那么对于正整数$n$,$l\mid n$的充要条件是$a^n\equiv 1\pmod m$

\textbf{2.} 若 $a$ 与 $m$ 互质,那么 $a^b\equiv a^{b\mod \varphi(m)}\pmod m$,因为

\begin{equation}
    a^b = a^{\varphi(m)\lfloor b/\varphi(m) + b\mod{\varphi(m)}}\equiv a^{b\mod{\varphi(m)}}\pmod m
    \nonumber
\end{equation}

若 $a$、$m$ 不互素,$a^b\equiv a^{b\mod \varphi(m) + \varphi(m)}\pmod m$

\textbf{3.} 若 $a$ 模 $m$ 的次数是 $\delta$,则 $\{a^0,a^1,\cdots,a^{\delta -1}\}$ 模 $m$ 不同余,注意这个集合可能不包含所有与$m$互素的元素(但是是一个子集),与$m$互素的元素数量为$\varphi(m)\ge \delta$,若$\delta = \varphi(m)$也就是说$a$模$m$的原根,这是模$m$的缩系

证明:假如有两个整数 $k, l$ 满足 $a^k\equiv a^l\pmod m,0\le k < l < \delta$,由于 $gcd(a,m)=1$,所以 $a^{l-k}\equiv 1\pmod m$,$l-k<\delta$,这与 $\delta$ 最小矛盾,原命题成立。

所以如果 $a$ 是一个 $m$ 的原根,那么 $\{a^0,a^1,\cdots,a^{\varphi(m)-1}\}$ 是一个缩系。因为$(a,m)=1$,所以$(a^i,m)=1$.

\textbf{4.} 若 $a$ 模 $m$ 的次数为 $r$, 设 $\lambda >0$,$a^{\lambda}$ 模 $m$的次数为$r/gcd(\lambda,r)$(简单的想只需要需要$r\mid \lambda t$,取最小的$t$)

证明:$r\mid \lambda t$,所以$\frac{r}{(\lambda,r)}\mid \frac{\lambda t}{(\lambda,r)}$,且$(\frac{r}{(\lambda,r)},\frac{\lambda}{(\lambda,r)})=1$,所以$\frac{r}{(\lambda,r)}\mid t$。又$(a^{\lambda})^{\frac{r}{(\lambda,r)}}\equiv (a^{r})^{\frac{\lambda}{(\lambda,r)}}\pmod m$,所以$t\mid \frac{r}{(\lambda,r)}$,所以$t=\frac{r}{(\lambda,r)}$

所以说如果$a$是$m$的一个原根(即$r=\varphi(m)$),那么$a^{\lambda}$是模$m$的原根的充要条件是$(\lambda,\varphi(m))=1$。如果找到了一个模$m$的原根$a$,对于任意$k$,只要$k$与$\varphi(m)$互质,那么$a^k$就是模$m$的一个原根。所以只要我们找到一个模$m$的原根,我们就可以找到其他所有原根。

\textbf{5.} $a$模$m$的次数是$s$,$b$模$m$的次数是$t$,若$(s,t)=1$,则$ab$模$m$的次数是$st$

\textbf{6.} $a$模$m$的次数是$\delta$,那么满足$(\lambda,\delta)=1,0<\lambda\le\delta$,且$a^{\lambda}$模$m$的次数为$\delta$,这样的$\lambda$有$\varphi(\delta)$个。由4.很容易证。

如果$a$是模$m$的原根,那么次数为$\varphi(m)$,那么如果$(\lambda,\varphi(m))=1$,那么$a^{\lambda}$是一个原根,这样的满足$(\lambda,\varphi(m))=1$的$\lambda$一共有$\varphi(\varphi(m))$个。

\textbf{7.} 若$\gcd(a,n)=1$,$r$是$a$模$n$的阶,那么$a^s\equiv a^t\pmod{n}$当且仅当$s\equiv t\pmod{r}$

\textbf{8.} 若$g$是模$n$的原根,那么$g^x\equiv g^y\pmod{n}$当且仅当$x\equiv y\pmod{\varphi(n)}$

\textbf{9.} $n$是一个有原根的正整数,假设$\gcd(a,n)=1$,则$x^k\equiv a\pmod{n} (k\ge 2) $有解当且仅当 

\begin{equation}
    a^{\varphi(n)/\gcd(k, \varphi(n))}\equiv 1\pmod{n}
\end{equation}

\textbf{10.} $p$是一个素数,$\gcd(a,p)=1$,$a$是一个$p$的$k$次剩余,当且仅当

\begin{equation}
    a^{(p-1)/\gcd(k,(p-1))}\equiv 1\pmod{p}
\end{equation}

\textbf{11.} 若$m$为$2,4,p^k,2p^k$($p$为奇素数)四者之一时,原根才存在。

 注:这里还有一些 Jacobi 符号在$k$次剩余上的拓展

\subsection{Carmichael's}

\textbf{1.Carmichael's 定理:}对任意的$a$,$a,n$为正整数,且互素,那么$a^{\lambda(n)}\equiv 1\pmod n$,$\lambda(n)$为 Carmichael 函数。$\lambda(n)$为满足该同余式的最小正整数,它总是小于等于$\phi(n)$,$\lambda(n)$ 的值称为 $a$ 模 $n$ 的 $\text{order}$.

\textbf{2.}

$$n=p_1^{r_1}p_2^{r_2}\cdots$$

那么$\lambda(n)=lcm(\lambda(p_1^{r_1}),\lambda(p_2^{r_2}),\cdots)$

\subsection{中国剩余定理}

\textbf{1.} 同余方程$a_1x_1+\cdots+a_nx_n+b\equiv 0\pmod m$有解的充要条件为$(a_1,\cdots,a_n,m)\mid b$,且模$m$情况下有$m^{n-1}(a_1,\cdots,a_n,m)$组不同的解(解所有的元都在模 $m$ 的剩余系下)。

\textbf{2.} $M=m_1\cdots m_n$,$m1,\cdots,m_n$两两互素,$f(x)=a_0+a_1x+\cdots+a_nx^n$。那么$x_0$是$f(x)\equiv 0\pmod M$的解当且仅当$x_0$是下面方程组的解

\begin{equation}
\begin{cases}
f(x)&\equiv 0\pmod {m_1} \\
&\cdots	\\
f(x)&\equiv 0\pmod {m_n}\\
\end{cases}    
\end{equation}

充分性比较显然,必要性由中国剩余定理可得(注意和$0$同余)。

\subsection{二次剩余}

\textbf{1.} $p$是一个奇素数,$a$不能被$p$整除,那么$x^2\equiv a\pmod{p}$要么没解,要么有两个解。

证明:若有$x^2\equiv y^2\equiv a\pmod{p}$,那么$p\mid (x+y)(x-y)$,所以$x\equiv y\pmod{p}$ or $x\equiv -y\pmod{p}$,这也是一个比较有意思的结论。

\textbf{2.} 二次剩余和二次非剩余的数量均为$(p-1)/2$

寻找二次剩余最简单的方法就是枚举$1^2,2^2,\cdots,\frac{(p-1)^2}{2},\frac{(p+1)^2}{2},\cdots,(p-1)^2\pmod{p}$,并且发现如果有解,$x^2\equiv (p-x)^2\pmod{p}$,即$x^2、(p-x)^2$指向同一个二次剩余,所以只需计算$x^2,1\le x\le (p-1)/2$。另外需要证明不存在$x,y\in[1,(p-1)/2]$指向同一个二次剩余。

\textbf{3.} 欧拉判别条件:$p$是一个奇素数,$\gcd(a,p)=1$,那么$a$是一个二次剩余当且仅当$a^{(p-1)/2}\equiv 1\pmod{p}$

由欧拉定理有

\begin{equation}
    (a^{(p-1)/2}-1)(a^{(p-1)/2}+1)\equiv a^{p-1} - 1\equiv 0 \pmod{0}
\end{equation}

所以$a^{(p-1)/2}\equiv 1\pmod{p}$或$a^{(p-1)/2}\equiv -1\pmod{p}$

证明:充分性,$a$是一个二次剩余,那么存在$x_0$使$x_0^2\equiv a\pmod{p}$,又由欧拉定理

\begin{equation}
    a^{(p-1)/2}\equiv (x^2)^{(p-1)/2}\equiv 1\pmod{p}
\end{equation}

必要性,假设$g$是模$p$的原根,那么存在$t$使$g^t\equiv a\pmod{p}$(原根构造缩系),那么

\begin{equation}
    g^{t(p-1)/2}\equiv a^{(p-1)/2}\equiv 1\pmod{p}
    \nonumber
\end{equation}

从而

\begin{equation}
    t(p-1)/2\equiv 0\pmod{p-1}
\end{equation}

所以t是一个偶数,所以

\begin{equation}
    (g^{t/2})^2\equiv g^t\equiv a\pmod{p}
\end{equation}

所以$a$是一个模$p$的二次剩余

\textbf{4.} 令$n=2^ep_1^{e_1}\cdots p_l^{e_l}$为n的素因子分解,假如$\gcd(a,n)=1$,那么$x^2\equiv a\pmod{n}$有解的充要条件为

\begin{itemize}
    \item 如果$e>1$(否则不考虑),若$e=2$,那么$a\equiv 1\pmod{4}$;若$e\ge 3$,那么$a\equiv 1\pmod{8}$
    \item 对任意的$i (1\le i\le k)$,都有$a^{(p_i-1)/2}\equiv 1\pmod{p_i}$
\end{itemize}


\subsection{Legendre 符号和 Jacobi 符号}

\noindent $p$是一个奇素数,$a$是一个整数,假设$\gcd(a,p)=1$,那么$Legendre\ symbol, \left(\dfrac{a}{p}\right)$,定义为

\begin{equation}
    \left(\frac{a}{p}\right)=
    \begin{cases}
    =1,&\text{如果$a$是模$p$的二次剩余}\\
    =-1,&\text{如果$a$是模$p$的非二次剩余}\\        
    \end{cases}
\end{equation}

需要注意,一般情况下,考虑$\gcd(a,p)=1$,如果$p$是$a$的因子,这个结果等于0,见英文版 wiki。

\textbf{1.} 由上面的欧拉判别条件有$\left(\dfrac{a}{p}\right)\equiv a^{(p-1)/2}\pmod{p}$,所以这是$Legendre\ symbol$形式的欧拉判别条件。

\textbf{2.} 若$a\equiv b\pmod{p}$,那么$\left(\dfrac{a}{p}\right)=\left(\dfrac{b}{p}\right)$,

\textbf{3.} $\left(\dfrac{ab}{p}\right)=\left(\dfrac{a}{p}\right)\left(\dfrac{b}{p}\right)$,由上面性质 1. 可以证明。这也表示这个符号是一个(完全)积性函数。

\textbf{4.} $\left(\dfrac{a^2}{p}\right)=1$,$因为x^2\equiv a^2\pmod{p}$。

由上面四条性质我们可以得到

\begin{itemize}
    \item 二次剩余的逆元依然是二次剩余,非二次剩余的逆元依然是非二次剩余
    \item 两个二次剩余或非二次剩余的积是二次剩余
    \item 二次剩余和非二次剩余的积是非二次剩余
\end{itemize}

\textbf{5.} $p$是一个奇素数,那么对所有整数

\begin{equation}
    \left(\frac{-1}{p}\right)=
    \begin{cases}
        1&\text{if $p\equiv 1\pmod{4}$}\\
        -1&\text{if $p\equiv 3\pmod{4}$}\\
    \end{cases}
\end{equation}

由$Legendre\ symbol$形式的欧拉判别条件很容易证明。这个结论可以用来判断一些特殊的二次剩余方程是否有解。


\subsubsection{高斯引理}

\textbf{1.} $p$为奇素数,假设$\gcd(a,p)=1$,$\omega$为

\begin{equation}
    \left\{1a,2a,3a,\cdots,\frac{p-1}{2}a\right\}
    \nonumber
\end{equation}

这里面模$p$为负数(或者大于$p/2$)的数的数量

\begin{equation}
    \left(\frac{a}{p}\right)=(-1)^{\omega}
\end{equation}

注:上面那个集合里面,两两不同余,它们模$p$落到一个完系中,而模$p$又一个这样的特殊的完系

\begin{equation}
    \left\{0, \pm 1, \pm 2,\cdots,\pm \frac{p-1}{2}\right\}
    \nonumber
\end{equation}

但是显然最上面那个集合里面的元素模$p$不可能等于 0,所以可能的结果会落到这个完系除 0 外的集合上。


\textbf{2.} 若$p$是奇素数,那么

\begin{equation}
    \left(\frac{2}{p}\right)=\left(-1\right)^{(p^2-1)/8}=
    \begin{cases}
        1, &\text{if }p\equiv \pm 1\pmod{8}\\
        -1,&\text{if }p\equiv \pm 3\pmod{8}\\
    \end{cases}
    \nonumber
\end{equation}

\subsubsection{二次互反律}

\textbf{1.} 如果$p,q$是两个不同的奇素数 

\begin{itemize}
    \item $\displaystyle \left(\frac{p}{q}\right)=\left(\frac{q}{p}\right)$, if one of $p,q\equiv 1~(\mathrm{mod}~4)$
    \item $\displaystyle \left(\frac{p}{q}\right)=-\left(\frac{q}{p}\right)$, if both $p,q\equiv 3(~\mathrm{mod}~4)$
    \item $\displaystyle \left(\frac{p}{q}\right)\left(\frac{q}{p}\right)=(-1)^{(p-1)(q-1)/4}$,把左边两项中任意一项直接移动到右边,也是成立的
\end{itemize}

\subsubsection{Jacobi 符号}

$a$是一个任意整数,$n$是一个正奇数,$n=p_1^{\alpha_1}\cdots p_k^{\alpha_k}$,那么$Jacobi$符号定义为

\begin{equation}
    \left(\frac{a}{n}\right)=\left(\frac{a}{p_1}\right)^{\alpha_!}\cdots \left(\frac{a}{p_k}\right)^{\alpha_k}
    \nonumber
\end{equation}

如果$\gcd(a,n)\neq 1$,那么上面的左边等于 0(左边是 Jacobi 符号,右边是 Legendre 符号)。

$m,n$是任意的正的奇数,$\gcd(a,n)=\gcd(b,n)=1$,那么

\begin{itemize}
    \item $a\equiv b~(\mathrm{mod}~n)$,那么$\displaystyle \left(\frac{a}{n}\right)=\left(\frac{b}{n}\right)$
    \item $\displaystyle \left(\frac{ab}{n}\right)=\left(\frac{a}{n}\right)\left(\frac{b}{n}\right)$
    \item $\gcd(m,n)=1$,那么$\displaystyle \left(\frac{a}{mn}\right)\left(\frac{a}{m}\right)=\left(\frac{a}{n}\right)$
    \item $\displaystyle \left(\frac{1}{n}\right)=1$
    \item $\displaystyle \left(\frac{-1}{n}\right)=(-1)^{(n-1)/2}$
    \item $\displaystyle \left(\frac{2}{n}\right)=(-1)^{(n^2-1)/8}$
    \item $\gcd(b,n)=1$,那么$\displaystyle \left(\frac{ab^2}{n}\right)=\left(\frac{a}{n}\right)$
    \item $\gcd(m,n)=1$,那么$\displaystyle \left(\frac{m}{n}\right)\left(\frac{n}{m}\right)=(-1)^{(m-1)(n-1)/4}$,或者 $\displaystyle \left(\frac{m}{n}\right)=(-1)^{(m-1)(n-1)/4}\left(\frac{n}{m}\right)$
\end{itemize}

对于最后一条性质的后面部分,如果$m$是偶数不能直接用,需要将 2 的幂提出来,之后再用。

需要注意,与$Legendre$符号不同的是,$\displaystyle \left(\frac{a}{n}\right)$如果结果为-1,那么$a\equiv x^2~(\mathrm{mod}~n)$可以确定无解,但是如果为-1,那么可能无解可能有解。

\subsection{连分数}

\begin{equation}
    [a_0,a_1,\cdots,a_n]=\frac{p_n}{q_n}
\end{equation}

\begin{equation}
    \begin{aligned}
    &p_0=a_0, p_1=a_1a_0+1,
    p_n=a_np_{n-1}+p_{n-2} &(2\le n\le N)\\
    &q_0=1,q_1=a_1,
    q_n=a_nq_{n-1}+q_{n-2} &(2\le n\le N)\\
    \end{aligned}
    \nonumber
\end{equation}


关于这个序列$a$,比如$a/b$的连分数序列,就是用欧几里得算法每一步迭代过程中的$a/b$(商)。

1. 任意有限简单连分数(finite simple continued fraction)一定可以表示为一个有理数。相反,任意有理数一定可以表示为一个有限简单连分数。

2. 任意无理数可以唯一的写成一个无限简单连分数,相反,一个无限简单连分数是一个无理数。

3. 任意具有周期性简单连分数是一个二次无理数(指的是整系数一元二次方程的根),相反,任意二次无理数标识为连分数具有周期性。

4. 一个非完全平方数的平方根的连分数是有周期的,并且若循环节以$a_0$开始,那么结尾一定是$2a_0$

\subsubsection{name}

1. $\sqrt{2}=[1,\overline{1}],~\sqrt{3}=[1,\overline{1,2}],~(\sqrt{5}-1)/2=[0,\overline{1}],~(\sqrt{5}+1)/2=[1,\overline{1}]$ 

\subsection{丢番图方程}

\subsubsection{Pell 方程}

设 $N$ 为非平方自然数,那么$x^2-Ny^2=1$有无穷自然整数解

\section{大整数分解}

\subsection{CFRAC, QS and NFS}

下面要介绍的几种分解方法基于这样一种重要的事实(起源于\href{https://en.wikipedia.org/wiki/Fermat%27s_factorization_method}{\color{blue}Fermat's method})

对于大整数 $N$,如果有两个整数$x,y$满足

\begin{equation}
    x^2\equiv y^2\pmod{N}, 0 < x < y < N, x\neq y, x+y\neq N
\end{equation}

那么$\gcd(x-y,N), \gcd(x+y,N)$就可能是$N$的非平凡因子。要找到上面的同余式,那么就需要找到

\begin{equation}
    \left(x_i=\prod{p_k^{e_k}}\right)\equiv \left(y_i=\prod{p_j^{e_j}}\right)\pmod{N}
\end{equation}

这样的同余式的集合,然后用乘法将同余符号两边凑成平方。

CFRAC, QS, NFS 三种方法的出发点都是去寻找

\begin{equation}
    x_k^2\equiv (-1)^{e_0}p_1^{e_{1k}}\cdots p_m{e_{mk}}\pmod{N}
\end{equation}

$p_i$ 是分解基($\text{FB}$),通过乘法凑出平方,即

\begin{equation}
    \sum_{1\leq k\leq n}\epsilon_k(e_{0k},\cdots,e_{mk})\equiv (0,\cdots, 0)\pmod{2}
\end{equation}

令$x=\prod_{1\leq k\leq n}x_{k}^{\epsilon_{k}}, y=(-1)^{v_0}p_1^{v_1}\cdots p_m^{v_m}$, 且$\sum_{k}\epsilon_k(e_{0k}, \cdots, e_{mk})=2(v_0,\cdots,v_m)$, 那么就有$x^2\equiv y^2~(\mathrm{mod}~N)$

\subsubsection{CFRAC}


\begin{equation}
    W_i=P_i^2-Q_i^2kN\Rightarrow P_i^2\equiv W_i \Leftrightarrow x_i^2\equiv (-1)^{e_0}\cdots p_m^{e_m}\pmod{N}
\end{equation}

$P_i/Q_i$是$\sqrt{kN}$的连分数逼近,$k$是任意选的一个数字,使$kN$不是一个平方数。


\subsubsection{Quadratic sieve}


定义

\begin{equation}
    Q(x)=(x+\left\lfloor\sqrt{N}\right\rfloor)^2-N
\end{equation}

那么

\begin{equation}
    Q(x) = (x+\left\lfloor\sqrt{N}\right\rfloor)^2-N\equiv (x+\left\lfloor\sqrt{N}\right\rfloor)^2\pmod{N}
\end{equation}

只需要$Q(x)$是一个平方数

二次筛法中,没有直接找$Q(x)$,而是先找一系列能被分解基完全分解的数(能被完全分解就成为光滑)。只需要寻找一系列$Q(x_i)$

\begin{equation}
    \sum_{i}Q(x_i)\equiv \sum_{i}(x_i+\left\lfloor\sqrt{N}\right\rfloor)^2\pmod{N}
\end{equation}

使得上面等式的左边是一个平方数,即在分解基下分解结果每个质因子的幂都是偶数。(本质是一个解线性方程的过程)

假如有个序列$X=\{0,\cdots,n\}$,对于分解基 F,我们想要求出所有关于分解基的光滑的$Q(X_i)$(在这之后在考虑线性组合的问题)。如果$Q(X_i)$是光滑的,对于每个分解基 F 中的质数p,下面的方程都需要满足。

\begin{equation}
    Q(X_i)\equiv (X_i+\left\lfloor\sqrt{N}\right\rfloor)^2-N\equiv 0\pmod{p}~\text{for each p}
\end{equation}

由于$p$都很小,所以方程很好解,得到$X_i\equiv a~(\mathrm{mod}~p)$,然后将$Q(X)$中$a,a+p,a+2p,\cdots$位置的数除以$p$(值得注意的是对于$p>2$,$X_i$有两个解)。这样得到的光滑数关于分解基中每个质数的幂都是 0 或者 1。那些$Q(X)$中值为 1 的数就对应一个光滑数。

然后就是一个解线性方程组的问题,即怎么组合这些光滑数分解基表达情况下,幂的组合是偶数。\href{https://zh.wikipedia.org/wiki/%E4%BA%8C%E6%AC%A1%E7%AF%A9%E9%81%B8%E6%B3%95}{\color{blue}wiki}给出了一个很好的例子 

\subsection{Polland's rho and p - 1 Methods}

\subsubsection{Polland's p - 1 Method}

如果能够找到一个与大合数$n$不互质的数$p$,那么可以直接求$\gcd(n,p)$求得$n$的一个因子。 


如果素因子 $p\mid n, n=pq$,那么$a^{p-1}\equiv 1 \mod{p}$,假如我们有$M=(p-1)x$(这里的$M$就是下面的$k$),那么对于满足$\gcd(a,p)=1$的$a$,$a^M=(a^{p-1})^x\equiv 1\pmod{p}$,那么$p\mid a^M-1$,又$p\mid n$,则有$p\mid\gcd(n,a^{M-1}-1)$。这样就可能找到一个$n$的平凡因子。一般$a$为2。

比如对于$k=\mathrm{lcm}(1,2,\cdots,B)$,$B$多大才比较合适呢。我们希望$\gcd(a^k-1,n)\neq 1$,对于$a^k-1$它的因子有$k$的所有素因子(不考虑特殊情况),那么如果$k$包含$n$的因子,那就很好。所以我猜测,$B$最好应该大于等于$n$所包含的最小因子。

下面是算法执行的步骤

\begin{itemize}
    \item 从 1 到 $n$ 中选择一个数 $a$,选择一个合适的$B$,$B$越大找到$N$因子的可能性越大,但算法需要的时间可能越久
    \item 计算$k$,$k$计算方式很多,比如可以是$\mathrm{lcm}(1,2,\cdots,B)$,可以是$B!$,又比如可以是
    \begin{equation}
            k=\prod_{p~prime, 1\le p \le B}p^{\left\lfloor\log B/\log p\right\rfloor}
    \end{equation}
    前面两种计算方式好像有缺点,比如$p-1$的分解包含一个素数的高次幂,那么就可能会失败。
    \item 计算$d=\gcd(a,n)$,$d\neq 1$,返回$d$
    \item 计算$e=\gcd(a^k-1,n)$,若$e=1$则增大$B$,若$e=n$,则重新选择$a$,否则返回$e$
\end{itemize}

\subsubsection{Polland's rho Method}

\subsection{Elliptic Curve Method}

\begin{equation}
    y^2=x^3+ax+b\pmod{p}
\end{equation}

$F_p$是一个定义在整数集合$\{0,1,2,\cdots,p-1\}$上的域,$a,b\in F_p$

\subsubsection{椭圆曲线}

对于椭圆曲线$y^2=x^3+ax+b$上的点$P=(x_p,y_p),Q=(x_q,y_q)$,定义$P+Q=R=(x_r,y_r)$,如图

\begin{figure}[h]
    \centering
    \includegraphics[width=0.5\textwidth]{../image/ECM.png}
\end{figure}


\begin{equation}
\begin{cases}
    \lambda = (y_p-y_q)/(x_p-x_q)~&\text{if $P\neq Q$}\\
    \lambda = (3x_p^2+a)/2y_p~&\text{if $P= Q$}\\
\end{cases}
\end{equation}

可以求得$R$

\begin{equation}
\begin{cases}
    x_r=\lambda^2-x_p-x_q\\
    y_r=\lambda(x_p-x_r)-y_p\\
\end{cases}
\end{equation}

如果是在有限域$F_p$情况下,那么就对所有操作取余。

\begin{equation}
    \begin{cases}
        \lambda = (y_p-y_q)/(x_p-x_q) \pmod{p}~ & \text{if $P\neq Q$}     \\
        \lambda = (3x_p^2+a)/2y_p\pmod{p}~   & \text{if $P=Q$} \\
    \end{cases}
\end{equation}

\begin{equation}
    \begin{cases}
        x_r=\lambda^2-x_p-x_q\pmod{p}    \\
        y_r=\lambda(x_p-x_r)-y_p\pmod{p} \\
    \end{cases}
\end{equation}


可以求得$R$

对于乘法$nP=P+(n-1)P$,如此不断递归,$2P=P+P$

\begin{itemize}
    \item 椭圆曲线上进行$Q-P$的减法运算,只需要将$P(x_p,y_p)$变为$-P(x_p,-y_p)$,然后做$Q$和$-P$的加法即可
    \item 在一个有限域$F_p$的情况下,单位元$O_E$定义为无穷远点,假设阶为$k$,那么对任意$P$,$kP=O_E$,且$P+O_E=P$
    \item 加法可以视作代数里面的乘法,$P$乘以常数$c$,可以视作代数里面对$P$取幂,$xP=D$可以视作$P^x= D$
\end{itemize}


\subsubsection{Elliptic Curve Method}

下面的所有运算都在$Z_p$中

\begin{enumerate}
    \item 随机取$k=B!$或$k=\mathrm{lcm}(2,3,5,7,\cdots)$
    \item 取$a,x,y\in Z_p$,计算$b=y^2-x^3-ax$. 重复 2. 直到$\gcd(4a^3+27b^2,n)\neq 1$. 这样就得到椭圆曲线 $E: y^2=x^3+ax+b$和其上一点$P(x,y)$
    \item 计算$kP$,每次计算,我们要么会得到一个新的点,要么会得到$n$的一个因子. 如果$kP\equiv \mathcal{O}_E\pmod{n}$ (这个的意思就是$\lambda$不存在,$\lambda$是一个求逆元的过程,所以实际上是逆元不存在),设此时$kP$的$\lambda$的分母是$m$,计算$\gcd(m,n)$,这个结果一定是$n$的非平凡因子
    \item 如果 3. 没有得到$n$的因子那么重复 2. (也可以进一步增大$B$)
\end{enumerate}

\begin{figure}[h]
    \centering
    \includegraphics[width=0.7\textwidth]{../image/ECM_example1.png}
\end{figure}

\section{离散对数}

给定$a,g,p\in\mathbb{N}$,计算满足$a^x\equiv g\pmod{p}$的$x$


$g$是$\mathbb{Z}_p$对应循环群的生成元,也就是原根,其阶数为$\varphi(p)$;$\mathbb{Z}_p$是一个循环群,当且仅当$p=1,2,4,p^k,2p^k(k>0)$

最朴素的方法就是

\begin{itemize}
    \item 计算$g^2(mod~p),g^3(mod~p)\cdots$
    \item 如果$g^k(mod~p)=a$就结束
    \item 复杂度是$p/2$
\end{itemize}

\subsection{Shanks' Baby-Step Giant-Step Algorithm}

令$s=\left\lfloor\sqrt{n}\right\rfloor$,令$k=im+j,i,j\in\{0,1\cdots,m-1\}$,由于$a\equiv g^k\pmod{p}\equiv g^{im+j}$,从而$g^j\equiv ag^{-im}\pmod{p}$;搜索$i,j$,找到满足$g^j\equiv ag^{-im}\pmod{p}$的$i,j$,然后就得到$x=im+j$

对于$y\equiv a^x\equiv~\mathrm{mod}~n$

\begin{itemize}
    \item $s=\left\lfloor n\right\rfloor$
    \item $y$就是给定的$g$,计算$(ya^r,r),r=0,\cdots,s-1$,$S=\{(y,0),(ya,1),(ya^2,2),\cdots,(ya^{s-1},s-1)~\mathrm{mod}~n\}$
    \item 计算$(a^{ts},ts),t=1,\cdots,s$,$T=\{(a^s,1),(a^{2s},2),(a^{s^2},s)~\mathrm{mod}~n\}$
    \item 将$S,T$按第一项排序,然后寻找$ya^r=a^{ts}$,计算$x=ts-r$即为答案$\log_a y~(\mathrm{mod}~n)$
\end{itemize}

复杂度$\sqrt{n}\log{n}$,用 hash 存储 $S$ 然后第二个表边计算边查询则是 $\sqrt{n}$,如果用 map 存储也是$\sqrt{n}\log{n}$。

\subsection{Silver-Pohlig-Hellman Algorithm}

假如$p$是一个大素数,然而$p-1$的素因子都相对较、小。

\begin{equation}
    a^x\equiv b\pmod{p}
\end{equation}

假如$GF(p)$的原根是$g$,那么$a\equiv g^i\pmod{p},b\equiv g^j\pmod{p}$,其中$i,j\in\{1,p-1\}$,带入上面的方程有$g^{xi}\equiv g^j\pmod{p}$,那么一定有$xi\equiv j\pmod{\varphi(p)=p-1}$,即$x=i^{-1}j\pmod{\prod_k p_k^{\alpha_k}}$,也就是如何解$i,j$,即$g^i\equiv a\pmod{p}$。

现在我们考虑$p-1=q^n$,假如

\begin{equation}
    x\equiv x_0+x_1q+\cdots+x_{n-1}q^{n-1}\pmod{q^n}
\end{equation}

\begin{equation}
    g^x\equiv a\pmod{p}
\end{equation}

那么


\begin{equation}
    \begin{aligned}
        a^{q^{n-1}}&\equiv(g^x)^{q^{n-1}}\\
        &=(g^{x_0+x_1q+\cdots+x_{n-1}q^{n-1}})^{q^{n-1}}\\
        &=(g^{q^{n-1}})^{x_0}\pmod{p}
    \end{aligned}
\end{equation}

从而转化为一个求$x_0$的离散对数问题,假如使用大步小步解出$x_0$后

\begin{equation}
    \begin{aligned}
        a^{q^{n-2}}&\equiv (g^x)^{q^{n-2}}\\
        &=(g^{x_0+x_1q+\cdots+x_{n-1}q^{n-1}})^{q^{n-2}}\\
        &=(g^{x_0q^{n-2}+ x_1q^{n-1}})\pmod{p}\\
        \Longleftrightarrow
        (g^{q^{n-1}})^{x_1}&\equiv ((g^{-x_0}a)^{q^{n-2}})\pmod{p}\\
    \end{aligned}
\end{equation}

这又是一个新的求$x_1$的离散对数问题,复杂度为$O(n\sqrt{q})$,以此类推求出$x_i (0\le i<n)$

对于$N=\varphi(p)=p-1=\prod_k p_k^{\alpha_k}$,我们需要求$x\equiv \log_a{b}\pmod{\varphi(p)=N=p-1}$,可以先求$x\equiv \log_a{b}\pmod{p}$,然后用 CRT。所以复杂度大约$\mathcal{O} (\sum_{k}\alpha_k\sqrt{p_k})$

下面给出解$a^x\equiv b\pmod{q}$算法的步骤,即解$x\equiv \log_a{b}\pmod{\varphi(q)}$

\begin{itemize}
    \item 分解$q-1=\prod_kp_k^{\alpha_k}$
    \item 对于分解结果中的因子$p^{\alpha}$,对于$x\equiv \log_a{b}\pmod{p^{\alpha}}$ (为方便,下标$k$均省略),假设
    \begin{equation}
        x\equiv x_0+x_1p+\cdots+x_{\alpha-1}p^{{\alpha-1}}\pmod{p^{\alpha-1}},~(0\le x_i < p)
    \end{equation}

    \begin{itemize}
        \item 预处理计算,$r_{p_i,j}=a^{j\cdot\frac{q-1}{p}}\pmod{q},0\le j\le p_i$
        \item 计算$x$
            \begin{itemize}
            \item $b_0=b$,寻找满足$b_1^{(q-1)/p}\equiv r_{p,j}\pmod{q}$的$j$,那么$x_0=j$
            \item $b_1=b\cdot a^{-x_0}$,寻找满足$b_1^{(q-1)/p^2}\equiv r_{p,j}\pmod{q}$的$j$,那么$x_1=j$
            \item $b_2=b\cdot a^{-x_0-x_1p}$,寻找满足$b_2^{(q-1)/p^2}\equiv r_{p,j}\pmod{q}$的$j$,那么$x_2=j$
            \item 类比,得到$x_i$
            \end{itemize}
        \item 通过 CRT 组合$x\equiv \log_a{b}\pmod{p_k^{\alpha_k}}$
    \end{itemize}
    
\end{itemize}

这里有两个具体的手算的例子,\href{https://blog.csdn.net/oampamp1/article/details/104061969}{\color{blue}例1}、\href{https://risencrypto.github.io/PohligHellman/}{\color{blue}例2}

下面给出枚举$r_{p,j}$的理由,假如$x_0,\cdots,x_{i-1}$已知

\begin{equation}
    \begin{aligned}
        (b)^{\frac{q-1}{p^{i+1}}}                              & \equiv (a^{x_0+\cdots+x_{\alpha-1}p^{{\alpha-1}}})^{\frac{q-1}{p^i}}                            \\
                                                               & =(a^{x_0+\cdots+x_{i-1}p^{{i-1}}})^{\frac{q-1}{p^{i+1}}}\cdot a^{x_i\cdot\frac{q-1}{p}}\pmod{q} \\
        \Longleftrightarrow
        (b\cdot a^{x_0+\cdots+x_{i-1}p^{i-1}})^{(q-1)/p^{i+1}} & \equiv a^{x_i\cdot\frac{q-1}{p}}\pmod{q}
    \end{aligned}
\end{equation}
然后需要用解$x_i$,所以需要预处理,这里的$a^{(q-1)/p}$相当于上面大步小步法里的$a$,当然最好使用大步小步解这个,但是看了一些博客包括书里面,预处理计算$r_{p,j}$枚举$j$都是 0 到 p,这是$\mathcal{O}(p)$的

需要注意,如果模数不是一个素数,可能也是可以的,这里关键的是我们需要使用费马小定理去掉底数$a$的指数上的多余的多项式

另外上面考虑的是对于$a^x\equiv b~(\mathrm{mod}~p)$,其中$a$恰好是模$p$的原根,这是最简单的情况,对于一般的情况下我们需要求出$a,b$用原根$g$的多少次幂表示(也就是最上面 4.2 说的$i,j$)。

\subsection{Index Calculus for Discrete Logarithms}

\href{https://www.csa.iisc.ac.in/~chandan/courses/CNT/notes}{\color{blue} Index Calculus Method}

这个链接里面的 lecture 21 是讲这个算法的。

需要求$a^x\equiv b\pmod{p}$,假设有一个平滑参数$y$,$p_1,\cdots,p_k$是小于$y$的素数,然后选取一个$\alpha \in[0,p-2]$,计算$a^{\alpha}\pmod{p}$,将这个结果用素因子表示,如果它能用小于$y$的素数完全分解,那就乘$a^{\alpha}$是$y-smooth$的,如果它不是$y-smooth$的就重新随机一个$\alpha$。假设$a^{alpha}\pmod{p}=\prod p_i^{e_i}$

\begin{equation}
    \alpha \equiv e_1 \log_a{p_1} +\cdots+e_{k}\log_a{p_k}\pmod{p-1}
\end{equation}

$\log_a{p_1},\cdots,\log_a{p_k}$都是不知道的,如果得到足够多这种关于$\log_a{p_i}$的方程(最好的情况

就是刚好$k$个),那么就可以解出来$\log_a{p_i}$。现在随机选择$\beta\in[0,p-2]$,计算$a^{\beta} b\pmod{p}$的结果,然后用小于$y$的素数分解得到

\begin{equation}
    \beta + \log_ab \equiv f_1log_a{p_1}+\cdots+f_k\log_k{p_k}\pmod{p-1}
\end{equation}

上面的等式里面只有$\log_ab$未知

\subsection{The Elliptic Curve Discrete Logarithm Problem}

在$P$的生成群$<P>$上,阶为$n$,给定$P、D$,求$x$使得$xP=D$,相当于解$x=\log_PD$

\subsubsection{Shanks’ Baby-Step Giant-Step Algorithm}

大步小步法一般简写$BSGS$,类似与$Z_p$中的$BSGS$

设

\begin{equation}
    x=s\left\lfloor n\right\rfloor + d, (0 < d < \left\lfloor n\right\rfloor) 
\end{equation}

那么$D-dP=s\left\lfloor n\right\rfloor P$,其中$d$是小步,$s$是大步,因为它成了一个关于$n$的系数,然后枚举$d和s$即可

\subsubsection{Silver-Pohlig-Hellman Algorithm}

将$xP=D$视作解$x=\log_PD\pmod{n}$的问题即可,点之间加法视作乘法,乘以常数$c$视作幂,一样的做法

这里有篇文章讲的比较好 \href{http://koclab.cs.ucsb.edu/teaching/ecc/project/2015Projects/Sommerseth+Hoeiland.pdf}{\color{blue} Pohlig-Hellman Applied in Elliptic Curve Cryptography}

\section{解同余方程}

\subsection{三次方根}

\begin{equation}
    x^3\equiv a\pmod{p}
\end{equation}

\begin{enumerate}
    \item 当$a=0$,解为$x\equiv 0\pmod{p}$
    \item 当$a\neq 0$
    \begin{enumerate}
        \item 若$p=2$,$a$一定等于 1,则解必定为$x\equiv 1\pmod{2}$
        \item 若$p$为奇素数,则$(a,p)=1$,设$p$的最小原根为g,则存在$t,s\in[0,\varphi(p)-1]$,满足
        \begin{align}
            a\equiv g^t\pmod{p}\\
            x\equiv g^s\pmod{p}\\
        \end{align}
        所以$3s\equiv t\pmod{\varphi(p)=p-1}$,该同余式未知数是$s$,$t$是求离散对数,有解的充要条件是$(3,p-1)\mid t$,解得$s$即可
    \end{enumerate}
\end{enumerate}

举例

\begin{equation}
    x^3\equiv 26\pmod{41}
\end{equation}

$g=6,\varphi(41)=40,t=17,s=19\Rightarrow x\equiv 34\pmod{41}$

\begin{equation}
    x^3\equiv 1\pmod{7}
\end{equation}

$g=3,\varphi(7)=6,t=0,s=0,2,4\Rightarrow x\equiv 1,2,4\pmod{7}$

$\textbf{这个计算方法适用于$n$次剩余}$

\subsection{平方根}

\begin{equation}
    x^2\equiv a\pmod{p}
\end{equation}

https://chenyangwang.gitbook.io/mathematical-base-for-information-safety/er-ci-tong-yu-shi-he-ping-fang-sheng-yu/mo-ping-fang-gen

https://oi-wiki.org/math/number-theory/quad-residue/

\href{https://blog.arpe1s.xyz/posts/2021/01/rth_root/}{link1}

\href{https://bbs.emath.ac.cn/thread-15822-1-1.html/
}{link2}

\subsubsection{模 4k + 3 平方根}

如果$p\equiv 3\mod{4}$

\begin{equation}
    \begin{aligned}
        \left(a^{(p+1)/4}\right)^2&\equiv a^{(p+1)/2}\\
        &\equiv x^{p+1}\\
        &\equiv x^2\cdot x^{p-1}\\
        &\equiv x^2\pmod{p}\\
    \end{aligned}
\end{equation}

所以$x\equiv a^{(p+1)/4}\pmod{p}$为一个解

\subsection{一些例子}

求解

\begin{align*}
    2x^7&\equiv 5\pmod{11}\\
    \Longleftrightarrow
    \log_2 2 + 7\log_2{x}&\equiv \log_25\pmod{10}\\
    \Longleftrightarrow
    1+7\log_2{x}&\equiv 4\pmod{10}\\
    \Longleftrightarrow
    \log_2x&\equiv 9\pmod{10}
\end{align*}

注意我们需要先求出$\log_2 2, \log_2 5$的值,模数很小的话可以直接枚举。

\section{超几何式}

\subsection{离散求和}

需要求

\begin{equation}
    \sum f(k) = g(k)+C
\end{equation}

如果有$\Delta g(k)=f(k)=g(k+1)-g(k)$

那么

\begin{equation}
    \sum f(k)=\sum f(k)\delta k=g(k)+C
\end{equation}

问题是如何求$g$ http://yyy.is-programmer.com/posts/205150.html

\section{特殊的数}

\subsection{斯特林数}

https://zhuanlan.zhihu.com/p/350774728

\section{裴蜀定理}

若 $(a,b)=1$,则 存在正整数 $x,y$ 使 $ax+by=1$

1. 若 $(a,b)=1$,则 $(a^i,b^j)=1$,对任意 $i,j$

\subsection{一些其他结论}

1. $\gcd(a_m,a_n)=a_{\gcd(m,n)}$

\section{一些site}

1. https://mathmu.github.io/MTCAS/Doc.html (计算机代数)


\section{Reference}

1. https://www.luogu.com.cn/blog/luogu/latex

2. [Carmichael function[卡迈克尔函数相关性质]](https://blog.csdn.net/AdijeShen/article/details/108476229)

3. http://gotonsb-numbertheory.blogspot.com/2014/06/primitive-roots.html

4. https://chaoli.club/index.php/2756/0(连分数入门)

5. https://crypto.stanford.edu/pbc/notes/contfrac/pell.html(Pell 方程)

6. https://math.uchicago.edu/~may/VIGRE/VIGRE2008/REUPapers/Yang.pdf (Pell 方程和连分数)

7. https://trizenx.blogspot.com/2018/10/continued-fraction-factorization-method.html (CFRAC code)

8. https://bbs.pediy.com/thread-224471.htm(二次筛法)

9. \href{https://mathmu.github.io/MTCAS/doc/IntegerFactorization.html#sec8}{整数因子分解}

\end{document}