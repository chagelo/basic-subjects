\documentclass{article}

% Language setting
% Replace `english' with e.g. `spanish' to change the document language
\usepackage[UTF8]{ctex}
% Set page size and margins
% Replace `letterpaper' with `a4paper' for UK/EU standard size
\usepackage[letterpaper,top=2cm,bottom=2cm,left=3cm,right=3cm,marginparwidth=1.75cm]{geometry}

% Useful packages
\usepackage{amssymb}
\usepackage{amsmath}
\usepackage{graphicx}
\usepackage{hyperref}
\usepackage{color}

\renewcommand{\theequation}{\arabic{section}.\arabic{equation}}
\newcommand{\setParDis}{\setlength {\parskip} {0.3cm} }
\newcommand{\nident}{\setlength{\parindent}{0pt}}



% \title{Your Paper}
% \author{You}

\begin{document}

\noindent \textbf{题1. }连分数分解 1711

取$k=1, N=1711, FB=\{-1,2,3,5\}$,则有

\begin{table}[h]
    \centering
    \begin{tabular}{llcc}
        \hline
        $i$ & $P_i$ & $W=P_i^2-kNQ_i^2$ & factorization\\
        \hline
        $0$ & $41$ & $-30$ & $(-1)\times 2 \times3\times 5$\\
        $1$ & $83$ & $45$ & $3^2\times 5$ \\
        $2$ & $124$ & $-23$ &$(-1)\times 23$\\
        $3$ & $331$ & $57$ & $3\times 19$\\
        $4$ & $455$ & $-6$ & $(-1)\times 2 \times 3$\\
        $5$ & $6246$ & $5$ & $5$\\
        $6$ & $100391$ & $-38$ & $(-1)\times 2 \times 19$\\
        $7$ & $207028$ & $9$ & $3^2$\\
        $8$ & $1756615$ & $-54$ & $(-1)\times 2\times 3^3$\\
      \end{tabular}
\end{table}

当$i=8$的时候,$455^2\times 1756615^2\equiv (2\times 3^2)^2\pmod{N}$,此时$(\gcd(799259843,N)$,\\$(799259807,N))=(59,29)$

\noindent \textbf{题2.} 

对于$N$,选择大于 2 的整数 A 构造 lucas 序列

\begin{equation}
  V_0 = 2, V_1 = A, V_j = AV_{j-1}-V_{j-2} \pmod{N}
\end{equation}

对于$M$,$M$ 是$p-\Bigl(\dfrac{A^2-4}{p}\Bigr)$的倍数,那么任意奇素数$p$一定能除尽$\gcd(N,V_{M}-2)$

我们需要使$\Bigl(\dfrac{A^2-4}{p}\Bigr)=-1$,即$A^2-4$是模$p$情况下的非二次剩余。

为了找到$p$,我们不断找$M$使得$\gcd(N,V_{M}-2)$不等于$1$或者$N$,就得到$N$的非平凡因子,所使用的$M$是$k!, k = 1, 2, 3\cdots$,我们有

\begin{equation}
  V_{k!}(A) = V_{k}\left(V_{k-1!}(A)\right)
\end{equation}


\noindent \textbf{题3.} 二次筛法分解 1046603 和 998771

$n = 1046603, \left\lceil\sqrt{1046603}\right\rceil=1024$时, factor base 为 

\begin{equation}
  Q(x) = (x + \left\lceil\sqrt{n}\right\rceil)^2-n\equiv (x + \left\lceil\sqrt{n}\right\rceil)^2\pmod{n}
\end{equation}

$P=50$情况下,找到了小于$50$的素数且素数$p$满足$N$是模$p$情况下的二次剩余,这样的$p$如下:

\begin{align*}
  \{2, 13, 17, 19, 29, 37, 41, 47\}
\end{align*}

将这些数作为分解基,去对满足$x\in[0,A=500]$的$Q(x)$进行筛选,最后可以得到形式为$p^2\equiv q\pmod{N}$的同余方程如下

\begin{equation}
  \begin{aligned}
    1030^2&\equiv 17\times 29^2\pmod{N}\\
    1319^2&\equiv 2\times 17\times 19\times 29\times 37\pmod{N}\\
    1370^2&\equiv 13^2\times 17^3\pmod{N}\\
    1493^2&\equiv 2\times 19\times 29^2\times 37\pmod{N} \\
  \end{aligned}
\end{equation}

从而

\begin{equation}
  \left(\gcd(1030\times1370\pm 13\times 17^2\times 29)\right) = (557,1879)
\end{equation}

同理$N=998771时$,得到分解基

\begin{equation}
  \{2, 5, 7, 11, 17, 19, 37, 43, 47\}
\end{equation}

最终可以得到

\begin{equation}
  \left(\gcd(1040039 \pm 16150, N)\right)=(1511, 661)
\end{equation}

\noindent \textbf{题4.} rho 算法分解

$f(x)=x^2+1,x_1=1,N=8051$情况下(题目给了$x_0=1$,为了计算方便,将$x_1$置为1)

\begin{equation}
  \begin{aligned}
    gcd(X[1] - X[1], N) &= 1\\
    gcd(X[3] - X[3], N)& = 1\\
    gcd(X[7] - X[6], N) &= 1\\
    gcd(X[7] - X[7], N) &= 1\\
    gcd(X[15] - X[12], N)& = 1\\
    gcd(X[15] - X[13], N)& = 1\\
    gcd(X[15] - X[14], N) &= 1\\
    gcd(X[15] - X[15], N) &= 1\\
    gcd(X[31] - X[24], N) &= 1\\
    gcd(X[31] - X[25], N) &= 83\\
  \end{aligned}
\end{equation}

所以

\begin{equation}
  N = 8051 = 83\times 97
\end{equation}

$f(x)=x^3+x+1,x_0=1,N=2701$情况下(题目给了$x_0=1$,为了计算方便,将$x_1$置为1)

\begin{equation}
  \begin{aligned}
    gcd(X[1] - X[1], N) &= 1\\
gcd(X[3] - X[3], N) &= 1\\
gcd(X[7] - X[6], N) &= 37\\
  \end{aligned}
\end{equation}

所以我们有

\begin{equation}
  N=2701=37\times73
\end{equation}

\noindent \textbf{题4.} 椭圆曲线分解$N=199843247$

当计算到$9624P$时,求斜率时,分母 dominator 模$N$逆元不存在,可以得到分母$\gcd(dom,N)$一定是$N$的一个非平凡因子,可以求得这个结果是 19423,从而

\begin{equation}
  N = 199843247 = 19423 \times 10289
\end{equation}

\noindent \textbf{题5.} $BSGS$分解求$x\equiv \log_{37}15\pmod{123}$

通过$BSGS$求得的两个列表如下

\begin{equation}
  \begin{aligned}
\{(15, 10), (24, 8), (27, 9), (63, 6), (117, 7)\}\\
    \{(1, 110), (10, 99), (16, 77), (37, 121), (100, 88)\}\\
  \end{aligned}
\end{equation}

对比两个列表发现找不到满足$ba^{is}$和$a^j$满足$ba^{is}\equiv a^j\pmod{N}$,所以判断出$37^{x}\equiv 15\pmod{123}$无解

\end{document}