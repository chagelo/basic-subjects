\documentclass{article}

% Language setting
% Replace `english' with e.g. `spanish' to change the document language
\usepackage[UTF8]{ctex}
% Set page size and margins
% Replace `letterpaper' with `a4paper' for UK/EU standard size
\usepackage[letterpaper,top=2cm,bottom=2cm,left=3cm,right=3cm,marginparwidth=1.75cm]{geometry}

% Useful packages
\usepackage{amssymb}
\usepackage{amsmath}
\usepackage{graphicx}
\usepackage{hyperref}

\renewcommand{\theequation}{\arabic{section}.\arabic{equation}}
\newcommand{\setParDis}{\setlength {\parskip} {0.3cm} }
\newcommand{\nident}{\setlength{\parindent}{0pt}}



% \title{Your Paper}
% \author{You}

\begin{document}

\setParDis
% \nident

\begin{equation}
    G(x)=a_0+a_1x+\cdots=\sum_{n\ge 0}a_nz^n
\end{equation}

$G(x)$称为$\{a_0,a_1\cdots\}$的生成函数

\section*{整数拆分}

用$1,5,10,25,50$硬币凑够$n$,那么

\begin{equation}
    G(x)=(1+x+x^2+\cdots)\cdots(1+x^{50}+x^{100}+\cdots)
\end{equation}

其中方案数为$x^n$前面的系数,$G(x)$的封闭形式为

\begin{equation}
    G(x)=\frac{1}{1-z}\frac{1}{1-z^5}\frac{1}{1-z^{10}}\frac{1}{1-z^{25}}\frac{1}{1-z^{50}}
\end{equation}

比较显然的我们有

\begin{equation}
    \begin{aligned}
    G(1)=G(1;x)&=(1+x+x^2+\cdots)=\frac{1}{1-x}\\
    G(2)=G(1,5;x)&=G(1;x)(1+x^5+x^{10}+\cdots)=G(1)\frac{1}{1-x^5}\\
    &\cdots\\
    G(5)=G(1,\cdots;x)&=G(4)(1+x^{50}+x^{100}+\cdots)=G(1,5,10,25;x)\frac{1}{1-x^{50}}\\
\end{aligned}
\end{equation}

我们可以发现比如$G(3)=G(2)(1-x^{25})$中$x^n$的系数$G_n(3)=G_n(2)-G_{n-25}(2)$,类似的

\begin{equation}
    \begin{aligned}
        G_n(2)&=G_{n-5}(2)+G_n(1)\\
        &\cdots     \\
        G_{n}(5)&=G_{n-50}(5)+G_{n}(4)\\
    \end{aligned}
\end{equation}

如果硬币不止上面几种,而包含$1,2,3\cdots$,那么

\begin{equation}
    G(x)=\frac{1}{1-x}\frac{1}{1-x^2}\frac{1}{1-x^3}\cdots
\end{equation}

$G(x)$中$x^n$前面的系数为方案数

\end{document}