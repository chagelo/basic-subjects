\documentclass{article}

% Language setting
% Replace `english' with e.g. `spanish' to change the document language
\usepackage[UTF8]{ctex}
% Set page size and margins
% Replace `letterpaper' with `a4paper' for UK/EU standard size
\usepackage[letterpaper,top=2cm,bottom=2cm,left=3cm,right=3cm,marginparwidth=1.75cm]{geometry}

% Useful packages
\usepackage{xcolor}
\usepackage{color}
\usepackage{amssymb}
\usepackage{manfnt}
\usepackage{amsmath}
\usepackage{graphicx}
\usepackage{listings}
\usepackage[colorlinks=true, allcolors=blue]{hyperref}

\title{科学社会主义在中国的实践与发展}
\author{刘远东 SA21011110}

\begin{document}

科学社会主义是马克思主义的三个组成部分之一,是马克思批判地继承了乌托邦社会主义中的合理因素而发展的理论,
是马克思批判地继承了乌托邦社会主义中的合理因素而发展的理论。它是马克思、恩格斯深入了解社会实践得出的,它揭示了人类社会发展规律、资本主义社会矛盾运动规律以及人类社会走向共产主义道路上的必经之趋势。

马克思恩格斯创立的科学社会主义,主要是建立在对发达的资本主义社会基本矛盾的分析之上,更多的是关于革命的理论和对世界历史发展的一般性规律。
对于经济文化落后的国家在取得社会主义革命胜利之后,如何建设社会主义,并没有给出答案。
十月革命给中国送来了马克思列宁主义,中国和俄国的国内情况类似,十月革命的胜利吸引着中国知识分子。
指导十月革命胜利的马克思主义,在当时看来,大概率是适合中国的。
五四运动之后,马克思主义逐渐在中国的思想文化领域发挥指导作用,指导中国的工人运动。
科学社会主义作为马克思主义思想体系的归宿和落脚点,为党和社会主义的建设与发展指明了方向。

新中国成立后,以毛泽东为核心的党中央领导全国人民进行了适合中国国情的社会主义道路的探索,为中国特色社会主义理论体系奠定理论基础。
1952年9月,毛泽东同志提出“中国怎样从现在逐步过渡到社会主义去”的指导方针和大致设想。
1953年,提出了“一化三改”,也就是,党在这个过渡时期的总路线和总任务,是要在一个相当长的时期内,逐步实现国家的社会主义工业化,并逐步实现国家对农业、对手工业和对资本主义工商业的社会主义改造。这条适合中国特点的社会主义改造道路,是中国共产党的独创性经验。
到1956年,全国绝大部分地区基本上完成了对生产资料私有制的社会主义改造,初步建立起公有制占绝对优势的社会主义经济制度,实现了中国历史上最深刻最伟大的社会变革,为当代中国一切发展进步奠定根本政治前提和制度基础。
中国有步骤地实现从新民主主义到社会主义的转变,迅速恢复了国民经济并开展了有计划的经济建设,在全国大部分地区基本上完成了对生产资料私有制的社会主义改造。中国共产党结合中国实际,在对农业、手工业和资本主义工商业进行社会主义改造的过程中,成功建立了社会主义经济制度,使中国进入了社会主义社会。
1956年,中国社会主义制度建立以后,以毛泽东为核心的党中央力图把社会主义的一般原则与中国的国情结合起来,对中国社会主义建设道路进行了艰难的探索,初步尝试了经济体制的改革,确定了建设社会主义的基本路线、方针和政策。
以毛泽东同志为核心的党的第一代中央领导集体依据科学社会主义原则,在中国初步探索并确立了社会主义制度。
邓小平同志强调,“我们搞改革开放,把工作重心放在经济建设上,没有丢马克思,没有丢列宁,也没有丢毛泽东。老祖宗不能丢啊!”江泽民同志告诫全党,“我们是干社会主义事业的,这应该成为广大党员、干部的理想信念和行动方向,真正入脑入心,做到不管遇到什么困难和风浪都毫不动摇。”胡锦涛同志指出,“我们说老祖宗不能丢,就是马克思主义基本原理不能丢,马克思主义科学世界观和方法论不能丢。”党的十八大以来,习近平总书记反复强调:“科学社会主义基本原则不能丢,丢了就不是社会主义。”
以上都是早期科学社会主义在中国的实践与发展,这个过程是曲折的。

文化大革命结束后,中国面临向何处去的艰难抉择。以邓小平同志为主要代表的中国共产党人,在深入总结经验教训基础上提出走自己的道路,建设有中国特色的社会主义,
作出把党的工作中心转移到经济建设上来、实行改革开放的历史性决策。
取得了举世瞩目的成就,创造性地探索出了一条有中国特色的社会主义现代化建设道路,形成了中国特色社会主义理论体系,这是马克思主义中国化的最新成果。
中国特色社会主义理论体系是对马克思列宁主义、毛泽东思想的继承与发展。它系统地回答了中国社会主义的根本任务、发展道路、发展阶段、发展动力、外部条件、政治保证、战略步骤、党的领导和依靠力量以及祖国统一等一系列基本问题,构成了比较完备的科学体系。

邓小平提出的社会主义本质论、社会主义初级阶段理论、社会主义市场经济理论和社会主义社会矛盾理论,
丰富和发展了科学社会主义的理论体系,创立了邓小平理论。这一理论是在新的世界历史条件下,在我国改革开放和现代化建设的实践中,
在总结我国社会主义胜利和挫折的历史经验并借鉴其他社会主义国家兴衰成败历史经验的基础上,逐步形成的。以邓小平为核心的党中央不断地总结历史经验,
不断地研究新情况、解决新问题,不断地开拓新局面、发展新理论,从而为中国特色社会主义理论的开创作出重大贡献。

以江泽民为核心的党的第三代领导集体,一直高举邓小平理论伟大旗帜,承前启后,继往开来,全面推进建设有中国特色社会主义的伟大事业,
并提出了三个代表重要思想。在领导中国人民进行改革和现代化建设的伟大实践中,对一系列关系到社会主义现代化建设全局的问题,进行了富有成效的思考和探索,
推动了我国社会主义现代化事业健康快速的发展。三个代表重要思想具有时代意义,它始终代表着中国先进生产力的发展要求,代表着中国先进文化的前进方向,
代表着中国最广大人民的根本利益。三个代表重要思想把中国特色社会主义理论推到了一个新的发展阶段。

以胡锦涛为核心的党的第四代领导集体,继续全面推进中国特色社会主义建设,提出了科学发展观,
进一步丰富和发展了中国特色社会主义理论,第一次明确提出“中国特色社会主义理论体系”的概念。科学发展观是我党以邓小平理论和“三个代表”重要思想为指导,
从新世纪、新阶段党和国家事业发展的全局出发提出的重要战略思想。科学发展观强调求真务实,是对中国特色社会主义理论体系精髓的发展;科学发展观强调全面协调可持续发展,
是对中国特色社会主义理论体系主题的发展;科学发展观提出了一系列紧密联系,相互贯通的新思想、新观点,丰富和发展了中国特色社会主义理论体系。

中国特色社会主义是马克思主义同中国具体实际相结合的产物,是科学社会主义在中国的运用和发展。
中国特色社会主义是科学社会主义历史发展的继续和进步,是科学社会主义在中国发展的新阶段,它给社会主义的发展增添了新的内容,
丰富了科学社会主义理论。中国特色社会主义的意义还在于打破了社会主义的传统模式,实现了社会主义体制的革新,使社会主义在中国的发展更符合中国的社会实际,为实践中的社会主义提供了许多重要的启示

在新的历史条件下,巩固和发展社会主义,是科学社会主义发展中的一个新课题。
中国特色社会主义在理论上丰富和发展了科学社会主义,也为社会主义在实践中的拓展提供了新的路径。坚持科学社会主义,就必须坚持用发展着的马克思主义指导新的实践,
使马克思主义中国化的最新成果成为推动中国社会不断发展进步的强大思想先导。

\end{document}

