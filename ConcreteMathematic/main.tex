\documentclass{article}

% Language setting
% Replace `english' with e.g. `spanish' to change the document language
\usepackage[UTF8]{ctex}
% Set page size and margins
% Replace `letterpaper' with `a4paper' for UK/EU standard size
\usepackage[letterpaper,top=2cm,bottom=2cm,left=3cm,right=3cm,marginparwidth=1.75cm]{geometry}

% Useful packages
\usepackage{amssymb}
\usepackage{amsmath}
\usepackage{graphicx}
\usepackage{hyperref}
\usepackage{color}

\renewcommand{\theequation}{\arabic{section}.\arabic{equation}}
\newcommand{\setParDis}{\setlength {\parskip} {0.3cm} }
\newcommand{\nident}{\setlength{\parindent}{0pt}}



% \title{Your Paper}
% \author{You}

\begin{document}

\setParDis
% \nident

\section{母函数与幂级数}

定义 

\begin{equation}
    \begin{aligned}
        F(x)&=\sum_{n=1}^{\infty}f_nx^n\\
        G(x)&=\sum_{n=1}^{\infty}g_nx^n
    \end{aligned}
\end{equation}

那么

\begin{equation}
    F(x)G(x)=\sum_{n=0}^{\infty}\Bigl(\sum_{k=0}^nf_kg_{n-k}\Bigr)x^n
\end{equation}

这个式子非常非常的有用

比如令$F(x)=G(x)=\sum_{n=0}x^n$,那么$F^2(x)=\sum_{n=0}^{\infty}\Bigl(\sum_{k=0}^n\Bigr)x^n=\sum_{n=0}^{\infty}nx^n$。

需要注意里面的卷积下标一定从 0 开始(因为是卷积),外面的不一定,比如可以定理 $F(x)=\sum_{n=1}x^n$。

1. $\Bigl(\sum_{n=0}x^n\Bigr)^2=\sum_{n=0}^{\infty}\Bigl(\sum_{k=0}^n\Bigr)x^n$

2. $\Bigl(\sum_{n=0}a_nx^n\Bigr)\Bigl(\sum_{n=0}x^n\Bigr)=\sum_{n=0}\Bigl(\sum_{k=0}^na_n\Bigr)x^n$

3. 

\begin{equation}
    \begin{aligned}
        \frac{1}{(1-x)^n}&=\sum_{n = 0}^{\infty}\binom{n+m-1}{m-1}z^n\\
        (1+x)^k&=\sum_{n=0}\binom{k}{n}x^n\\
        \frac1{\sqrt{1-4x}}&=\sum_{n=0}\binom{-\frac12}{n}(-4x)^n=\sum_{n=0}\binom{2n}{n}x^n\\
    \end{aligned}
\end{equation}

所以就会有很多有意思的结论,比如 $1/\sqrt{1-4x}=\sum_{n=0}^{\infty}\binom{-\frac12}{n}(-4x)^n$

\section{一些site}

1. https://mathmu.github.io/MTCAS/Doc.html (计算机代数)


\section{Reference}


9. \href{https://mathmu.github.io/MTCAS/doc/IntegerFactorization.html#sec8}{整数因子分解}

\end{document}